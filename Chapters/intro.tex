\chapter{Introduction}
\label{chapter:intro}
This thesis is concerned with the parsing as deduction paradigm, as orchestrated by type-logical grammars, perceived through the lens of a data-driven experimental setting.
It seeks to bridge the gap between formal theory and empirical practice, integrating insights from half a century of progress in categorial type logics with recent advances in neural networks and natural language processing.

The key theme and the underlying goal behind this work is the development of a concrete, robust and widely applicable methodology for syntactic analysis that naturally lends itself to semantic uses.
The path towards this goal includes many twists and turns, forcing study of semantics to assume a secondary role throughout the thesis.
What is primarily addressed instead are exactly these twists and turns, the resolution of which brings us at an arm's reach of semantics by this thesis' conclusion.

Even though the methods applied are language-agnostic and highly general, the experiments performed focus on Dutch.
Dutch is a particularly challenging language to work with, owing to its many syntactic variations with respect to word order.
As such, it provides an excellent testing ground that puts our designs and hypotheses under rigorous tests, while soliciting novel approaches and creative solutions.

The thesis is organized in four major chapters.
Each chapter is largely autonomous, in the sense that it treats a different topic and seeks to answer a different research question.
However, there is a weak linear dependency between chapters as each one progressively expands upon its predecessors' results; thus, an exhaustive reading should best follow the document in the sequential order it is presented.
Chapters all share a similar structure; they have their own brief introductory overview and a summarizing conclusion which aims to concisely break down its scope and contribution.
A rapid but informative reading could be done on the basis of these summaries.

We begin in Chapter~\ref{chapter:tlg} by providing a brief account of Type-Logical Grammars and their most common varieties.
In considering Dutch, we notice practical issues caused by the language's peculiarities and seek out ways to address them.
The chapter's motif is the balance between formal rigor and pragmatic applicability of type-logical grammars for large-scale use in a language like Dutch.
The chapter concludes with a heedful compromise that retains the best of both worlds.

Having specified the grammar and its latent logic, Chapter~\ref{chapter:extraction} then sets out to populate a data-driven type lexicon.
The dependency-annotated sentences of the written Dutch corpus, Lassy-Small, is used to extrapolate type-logical derivations and phrasal type assignments.
The algorithmic process of this conversion is detailed, alongside the transformations necessitated by incompatibilities between the corpus' annotation philosophy and the grammar's specifications.

Chapter~\ref{chapter:sup} makes for a change of pace; it deals with supertagging, the statistical learning process through which type sequences may be assigned to sentences not included in the source corpus.
In reviewing the extracted type system, it notes a distinction to prior supertagging applications, related to the significantly larger size of the lexicon.
The question then turns to designing a system capable of overcoming this complication, accomplished through a simple reformulation of the problem statement.

Finally, Chapter~\ref{chapter:parsing} seeks to alleviate the proof-theoretic concessions made during the grammar's specification.
The topic revolves around the manipulation of an ambiguous type-logical proof structure utilizing proof-external information sources such as preferential biases exerted by semantic content.

A high-level synopsis plus a few concluding remarks are presented in Chapter~\ref{chapter:conclusion}.